A administra��o da empresa cliente detectou abusos aos recursos de impress�o
por parte dos usu�rios, estes abusos est�o gerando v�rios problemas, como
por exemplo, gasto excessivo de papel e tinta.

Na tentativa de conscientizar os usu�rios, foram realizadas campanhas de
preserva��o ambiental focando as �rvores (que originam as mat�rias-primas
do papel), mas n�o obtiveram sucesso.

O fracasso dessas iniciativas levaram a constata��o de que seria necess�rio
criar uma pol�tica de controle das impress�es, a fim de verificar quando
e quais usu�rios est�o utilizando as impressoras e se os documentos sendo
impressos fazem parte do foco do neg�cio. Desenvolver e implantar essa
pol�tica consumiria muito tempo e recursos do Departamento de Tecnologia da
Informa��o, e a diretoria considerou a contrata��o de servi�os terceirizados
para fazer a gest�o das impress�es, por�m, seu desconhecimento sobre o
mercado de \Outsourcing\  de impress�o tornou necess�rio realizar um estudo
de viabiliza��o em curto prazo e de baixo custo.
